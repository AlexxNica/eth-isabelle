\documentclass[11pt,a4paper]{article}
\usepackage{isabelle,isabellesym}

% further packages required for unusual symbols (see also
% isabellesym.sty), use only when needed

%\usepackage{amssymb}
  %for \<leadsto>, \<box>, \<diamond>, \<sqsupset>, \<mho>, \<Join>,
  %\<lhd>, \<lesssim>, \<greatersim>, \<lessapprox>, \<greaterapprox>,
  %\<triangleq>, \<yen>, \<lozenge>

%\usepackage{eurosym}
  %for \<euro>

%\usepackage[only,bigsqcap]{stmaryrd}
  %for \<Sqinter>

%\usepackage{eufrak}
  %for \<AA> ... \<ZZ>, \<aa> ... \<zz> (also included in amssymb)

%\usepackage{textcomp}
  %for \<onequarter>, \<onehalf>, \<threequarters>, \<degree>, \<cent>,
  %\<currency>

% this should be the last package used
\usepackage{pdfsetup}

% urls in roman style, theory text in math-similar italics
\urlstyle{rm}
\isabellestyle{it}

% for uniform font size
%\renewcommand{\isastyle}{\isastyleminor}


\begin{document}

\title{Formal Verification of Deed contract in Ethereum Name Service}
\author{Yoichi Hirai\footnote{\texttt{i@yoichihirai.com}}}
\maketitle

\tableofcontents

% sane default for proof documents
\parindent 0pt\parskip 0.5ex

\section{Introduction}

This document describes a formal verification result about one contract in
an Ethereum Name Service implementation.  The verified contract is relatively small,
but this is the first ``real''\footnote{The word ``real'' means it's
produced by the Solidity compiler and aimed for production.}
contract that I have analyzed.

This has some issues (the EVM implementation is not tested against
others!), but I'm publishing this already because this project makes a
good example on the amount of work (and the level of detail) required to
verify a smart contract using machine-assisted logical inference.  At this
point already, if I were to
implement a smart contract that holds more than 100k dollars, and if
I'm in charge of controlling the schedule, I would consider this kind of
development (the other option is try-small first).

\subsection{Which Smart Contract is this about}

The target of the verification is the \texttt{Deed} contract,
which can be found here. %TODO: turn this into a link
The current development uses the bytecode obtained from the
Solidity compiler version
\texttt{2d9109ba453d49547778c39a506b0ed492305c16}. %TODO turn this
                                %into a link

\subsection{What is Proven}

The proven property is about one invocation of the Deed contract.
In short, ``only the registrar can decrease the balance.''
The invocation can be deep into the nested reentrancy calls, but pick
one of these.
There are assumptions and there are conclusions.

Assumptions:
\begin{itemize}
  \item the account has the bytecode of the Deed contract or the
    account has no code;
  \item the caller does not have the address stored at index~0 of the
    account's storage (i.e. is not the registrar);
  \item the account's balance and the sent value added together do not
    overflow the range of 256-but unsigned integers;
  \item the account is not marked as killed when the account is called.
\end{itemize}

Conclusions:
\begin{itemize}
  \item the account's balance after the call is not smaller than the
    account's balance before the call;
  \item the account is not marked as killed after the call.
\end{itemize}


\subsection{What can Go Wrong}

The analysis cuts corners but it is reasonablly equipped for safety
properties (i.e. ``certain changes never happen on the account
state'').  The analysis considers reentrancy, the account erasure after execution
of the \texttt{SUICIDE} opcode, the byte level organization of EVM
memory and storage, and the fact that the balance of an account can
increase even when the code of the account is not invoked.
The analysis is not aware of out-of-gas failures and the stack depth
failures, but the analysis does not miss any kind of account state
changes because of that.

The biggest pitfall currently is the EVM implementation in
Isabelle/HOL.  Although this is a new EVM implmentation, it is not
tested against the standard EVM tests!  This is already wrong.
So, after getting this document in shape, the next thing I try is to
test the new EVM implementation against the standard EVM tests.

Still the new EVM implementation was worth creating as shown below.
It allows us to reason about all possible executions!

\subsection{Links}

To get updates on similar projects, follow \url{http://gitter.im/ethereum/formal-methods}.

% generated text of all theories
\input{session}

% optional bibliography
%\bibliographystyle{abbrv}
%\bibliography{root}

\end{document}

%%% Local Variables:
%%% mode: latex
%%% TeX-master: t
%%% End:
